\section{Additional information}

\begin{table}[h]
  \begin{center}
    \centering
    \begin{tabular}{@{}lllll@{}}
      \toprule
      Name        & Author              & URL (if available) \\ \midrule
      Tetris      & Felix Schoeller     &             -  \\
      Raytracer   & Alex Quach          & https://github.com/aquach/from-nand-to-raytracer   \\
      GASchunky   & Gavin Stewart       & https://github.com/gav-/Nand2Tetris-Games\_and\_Demos  \\
      GABoing     & Gavin Stewart       & https://github.com/gav-/Nand2Tetris-Games\_and\_Demos  \\
      Hackenstein & Quester Zen         & https://github.com/QuesterZen/hackenstein3D  \\
      Doom        & Jona Leon Heywinkel &             -  \\
      *Raycaster  & Julius Armbrüster   &             -  \\
      Axis        & Markus Brenneis     &             -  \\
      Brainhack   & Markus Brenneis     &             -  \\
      Minesweeper & Patrick Müller      &             -  \\ \bottomrule
    \end{tabular}
    \small
    \item The project by Julius Armbrüster did not have a name, so Raycaster was chosen to describe it
    \caption{The VM programs tested on the new emulator implementation.}%
    \label{table:tested}
  \end{center}
\end{table}

% Hier können Sie Ihren Anhang definieren.

% Achten Sie darauf, dass der Anhang in Ihrer \texttt{thesis.tex}
% initial auskommentiert ist.
% Der entsprechende Part befindet sich nahe dem Ende der Datei.
% Entfernen Sie bei Bedarf die Kommentierung um den Anhang nutzen zu können.


% \section{Nutzung}

% Der Anhang wird wie die Abschnitte des Hauptteils der Arbeit gestaltet,
% also mit \texttt{\textbackslash section} Befehlen.

% \subsection{Unterabschnitte}
% Die Verwendung von Unterabschnitten im Anhang
% mittels \texttt{\textbackslash subsection}
% funktioniert ebenfalls!
