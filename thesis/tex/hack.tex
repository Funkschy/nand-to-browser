\section{Nand to Tetris and the Hack Architecture}
\subsection{The sections of Nand to Tetris}
\begin{itemize}
  \item Chips und Logic Gates (nicht Teil der Arbeit)
  \item CPU und Assembly
  \item Virtuelle Machine
  \item High level Sprache und Betriebssystem (nicht Teil der Arbeit)
\end{itemize}

\subsection{How does the Hack VM work}
\subsubsection{Example: Adding numbers in a Loop}
\begin{lstlisting}[language=C, caption={Calculate 1 + 2 + 3 in C}, captionpos=b]
  int i = 1;
  int sum = 0;
  while (i <= 3) {
    sum += i;
    i++;
  }
\end{lstlisting}

\subsection{Hack Bytecode}
\begin{lstlisting}[caption={Calculate 1 + 2 + 3 in the Hack VM}, captionpos=b]
  // i = 1
  push constant 1
  pop local 0

  // sum = 0
  push constant 0
  pop local 1

  label LOOP_START

  // the Hack bytecode does not have an <= instruction,
  // therefore use i < 4 instead of i <= 3
  push local 0
  push constant 4
  lt

  // if i >= 4 jump out of the loop
  not
  if-goto LOOP_END

  // sum = sum + i
  push local 1
  push local 0
  add
  pop local 1

  // i = i + 1
  push local 0
  push constant 1
  add
  pop local 0

  // jump to the beginning of the loop again
  goto LOOP_START
  label LOOP_END
\end{lstlisting}
