\section{Nand to Tetris und die Hack Architektur}
\subsection{Die Abschnitte des Nand to Tetris Kurses}
\begin{itemize}
  \item Chips und Logic Gates (nicht Teil der Arbeit)
  \item CPU und Assembly
  \item Virtuelle Machine
  \item High level Sprache und Betriebssystem (nicht Teil der Arbeit)
\end{itemize}

\subsection{Ein Überblick über VM Architektur im Allgemeinen}
\subsubsection{Beispiel: Zahlen in Schleife addieren in simpler VM}
\begin{lstlisting}[language=C, caption={Berechne 1 + 2 + 3 in C}, captionpos=b]
  int i = 1;
  int sum = 0;
  while (i <= 3) {
    sum += i;
    i++;
  }
\end{lstlisting}
\begin{lstlisting}[caption={Berechne 1 + 2 + 3 in einer Stack-basierten VM}, captionpos=b]
  // i = 1
  push 1
  pop i

  // sum = 0
  push 0
  pop sum

  label LOOP_START
  push i
  push 3
  // i <= 3 pusht entweder true oder false auf den stack
  less-than-or-equal
  // wenn der vorherige check false war, springen wir aus der Schleife
  goto-if-false LOOP_END

  // wenn i immer noch <= 3, addieren wir zuerst i und sum
  // und schreiben das Ergebnis der Addition wieder in die Summe
  push i
  push sum
  add
  pop sum

  push i
  push 1
  add
  pop i  // i = i + 1

  // Springe zum Anfang der Schleife
  goto LOOP_START

  label LOOP_END
\end{lstlisting}

\subsection{Hack Bytecode}
\subsubsection{Das obige Beispiel in der Hack Architektur}
\begin{lstlisting}[caption={Berechne 1 + 2 + 3 in der Hack VM}, captionpos=b]
  // i = 1
  push constant 1
  pop local 0

  // sum = 0
  push constant 0
  pop local 1

  label LOOP_START

  // die Hack Architektur besitzt keine <= Instruktion,
  // insofern benutzen wir i < 4 anstatt i <= 3
  push local 0
  push constant 4
  lt

  // falls i >= 4 ist springen wir aus der Schleife
  not
  if-goto LOOP_END

  // sum = sum + i
  push local 1
  push local 0
  add
  pop local 1

  // i = i + 1
  push local 0
  push constant 1
  add
  pop local 0

  // Springe zum Anfang der Schleife
  goto LOOP_START
  label LOOP_END
\end{lstlisting}
