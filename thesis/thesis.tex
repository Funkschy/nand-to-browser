%%%%%%%%%%%%%%%%%%%%%%%%%%%%%%%%%%%%%%%%%%%%%%%%%%%%%%%%%%%%%%%%%%%%%%%%%%%%%%%%
% Universität Düsseldorf                                                       %
% Lehrstuhl für Softwaretechnik und Programmiersprachen                        %
% Vorlage für Bachelor- und Masterarbeiten                                     %
% Erstellt: 2019-09-03                                                         %
%%%%%%%%%%%%%%%%%%%%%%%%%%%%%%%%%%%%%%%%%%%%%%%%%%%%%%%%%%%%%%%%%%%%%%%%%%%%%%%%
\documentclass{hhuthesis}


%%%%%%%%%%%%%%%%%%%%%%%%%%%%%%%%%%%%%%%%%%%%%%%%%%%%%%%%%%%%%%%%%%%%%%%%%%%%%%%%
%% Einstellungen zur Personalisierung                                         %%
%%                                                                            %%
%% Im Folgenden können Sie Ihre Arbeit personalisieren.                       %%
%%%%%%%%%%%%%%%%%%%%%%%%%%%%%%%%%%%%%%%%%%%%%%%%%%%%%%%%%%%%%%%%%%%%%%%%%%%%%%%%

%% Spracheinstellung
%% Kommentieren Sie die entsprechende Zeile ein bzw. aus.
%% Wir empfehlen jedem sich an einer englischen Arbeit zu versuchen.
\usepackage[ngerman,english]{babel} % English
% \usepackage[english,ngerman]{babel} % Deutsch

%% Ihr Name
\author{Felix Schoeller}

%% Der Titel der Arbeit
\title{VM development targeting the web browser with Rust and WebAssembly}

%% Der zu erreichende Abschluss, entweder Bachelor oder Master
\gratuationtype{Bachelor}
% \gratuationtype{Master}

%% Beginn- und Abgabedaten der Arbeit
\begindate{17. September 2022} % Beginn
\duedate{17. Dezember 2022} % Abgabe

%% Erst- und Zweitgutachter
\firstexaminer{Dr.~John~Witulski}
\secondexaminer{Prof.~Dr.~Michael~Schöttner}

%% Farb- oder Schwarzweißdruck
% Benutzen Sie das Kommando \blackwhiteprint,
% wenn sie in schwarzweiß drucken möchten.
% Im Farbdruck ist jede farbige Seite idR teurer.
% \blackwhiteprint  % Kommentarzeichen entfernen für Schwarzweißdruck

%%%%%%%%%%%%%%%%%%%%%%%%%%%%%%%%%%%%%%%%%%%%%%%%%%%%%%%%%%%%%%%%%%%%%%%%%%%%%%%%
%% (Ende) Einstellungen zur Personalisierung                                  %%
%%%%%%%%%%%%%%%%%%%%%%%%%%%%%%%%%%%%%%%%%%%%%%%%%%%%%%%%%%%%%%%%%%%%%%%%%%%%%%%%
%% LaTeX Packages in Nutzung                                                  %%
%%                                                                            %%
%% Im folgenden können Sie für die Niederschrift Ihrer Arbeit benötigte       %%
%% LaTeX-Pakete einbinden.                                                    %%
%% Diese Vorlage kommt bereits mit einigen nützlichen inkludierten Paketen.   %%
%%%%%%%%%%%%%%%%%%%%%%%%%%%%%%%%%%%%%%%%%%%%%%%%%%%%%%%%%%%%%%%%%%%%%%%%%%%%%%%%

%% Macht den \todo-Befehl verfügbar.
%% Hiermit können Sie Abschnitte annotieren,
%% welche weiterer Bearbeitung bedürfen.
\usepackage[textsize=scriptsize]{todonotes}

%% Zeige Zeilennummern in der Arbeit an.
%% Der \linenumbers Befehl muss hierzu aufgerufen werden.
%% Praktisch für Feedback Ihrer potentiellen Korrekturleser!
\usepackage{lineno}
% \linenumbers % <- Kommentar entfernen!


%% Häufig benutzte mathematische Packages.
\usepackage{amsfonts}
\usepackage{amsmath}
\usepackage{amssymb}


\usepackage{listings} % Einbindung von Code
\usepackage{algorithmicx} % Angabe von Algorithmen in Pseudocode
\usepackage{siunitx} % \num Befehl zum einfacheren Formatieren von Zahlen.
\usepackage{enumitem} % Leichter konfigurierbare enumerate-Umgebungen.
\usepackage{subcaption} % Unterteilung von Figures in Subfigures.
\usepackage{hyperref} % Klickbare Referenzen (z.B. im Inhaltsverzeichnis)
\usepackage{url} % \url Kommando für Darstellung von Links
\usepackage{csquotes} % Improved quoting.
\usepackage{xspace} % Nicht terminierte Kommandos essen keinen Whitespace mehr.

%% Tabellen
\usepackage{tabularx} % tabularx Umgebung für mehr Kontrolle über Tabellen.
\usepackage{booktabs} % \toprule, \midrule, \bottomrule
\usepackage{multirow}
\usepackage{multicol}
\usepackage{longtable} % Große Tabellen gehen über mehrere Seiten.

%% Intelligenteres Referenzieren mittels \cref.
%% \languagename um dynamisch zwischen ngerman oder english zu wechseln.
\usepackage[\languagename,capitalize]{cleveref}

%%%%%%%%%%%%%%%%%%%%%%%%%%%%%%%%%%%%%%%%%%%%%%%%%%%%%%%%%%%%%%%%%%%%%%%%%%%%%%%%
%% (Ende) LaTeX Packages in Nutzung                                           %%
%%%%%%%%%%%%%%%%%%%%%%%%%%%%%%%%%%%%%%%%%%%%%%%%%%%%%%%%%%%%%%%%%%%%%%%%%%%%%%%%

%%%%%%%%%%%%%%%%%%%%%%%%%%%%%%%%%%%%%%%%%%%%%%%%%%%%%%%%%%%%%%%%%%%%%%%%%%%%%%%%
%% Custom Syntax highlighting                                                 %%
%%%%%%%%%%%%%%%%%%%%%%%%%%%%%%%%%%%%%%%%%%%%%%%%%%%%%%%%%%%%%%%%%%%%%%%%%%%%%%%%
\usepackage{xcolor}

\definecolor{commentsColor}{rgb}{0.497495, 0.497587, 0.497464}
\definecolor{keywordsColor}{rgb}{0.000000, 0.000000, 0.635294}
\definecolor{stringColor}{rgb}{0.558215, 0.000000, 0.135316}

\lstset{ %
  backgroundcolor=\color{white},   % choose the background color; you must add \usepackage{color} or \usepackage{xcolor}
  basicstyle=\footnotesize,        % the size of the fonts that are used for the code
  breakatwhitespace=false,         % sets if automatic breaks should only happen at whitespace
  breaklines=true,                 % sets automatic line breaking
  captionpos=b,                    % sets the caption-position to bottom
  commentstyle=\color{commentsColor}\textit,    % comment style
  deletekeywords={...},            % if you want to delete keywords from the given language
  escapeinside={\%*}{*)},          % if you want to add LaTeX within your code
  extendedchars=true,              % lets you use non-ASCII characters; for 8-bits encodings only, does not work with UTF-8
  % frame=tb,	                     % adds a frame around the code
  keepspaces=true,                 % keeps spaces in text, useful for keeping indentation of code (possibly needs columns=flexible)
  keywordstyle=\color{keywordsColor}\bfseries,       % keyword style
  language=Python,                 % the language of the code (can be overrided per snippet)
  otherkeywords={*,...},           % if you want to add more keywords to the set
  numbers=left,                    % where to put the line-numbers; possible values are (none, left, right)
  numbersep=5pt,                   % how far the line-numbers are from the code
  numberstyle=\tiny\color{commentsColor}, % the style that is used for the line-numbers
  rulecolor=\color{black},         % if not set, the frame-color may be changed on line-breaks within not-black text (e.g. comments (green here))
  showspaces=false,                % show spaces everywhere adding particular underscores; it overrides 'showstringspaces'
  showstringspaces=false,          % underline spaces within strings only
  showtabs=false,                  % show tabs within strings adding particular underscores
  stepnumber=1,                    % the step between two line-numbers. If it's 1, each line will be numbered
  stringstyle=\color{stringColor}, % string literal style
  tabsize=2,                       % sets default tabsize to 2 spaces
  title=\lstname,                  % show the filename of files included with \lstinputlisting; also try caption instead of title
  columns=fixed                    % Using fixed column width (for e.g. nice alignment)
}

\lstdefinelanguage{Hack}{
  keywords={add, sub, eq, gt, lt, and, or, not, neg, push, pop,
    if-goto, goto, label, function, call, return},
  ndkeywords={argument, local, static, constant, this, that, pointer, temp},
  ndkeywordstyle=\color{darkgray}\bfseries,
  sensitive=false, % keywords are not case-sensitive
  alsoletter=-,
  morecomment=[l]{//}, % l is for line comment
  morecomment=[s]{/*}{*/}, % s is for start and end delimiter
  morestring=[b]" % defines that strings are enclosed in double quotes
}
\lstdefinelanguage{Rust}{
  keywords={as,break,const,continue,crate,else,enum,extern,false,fn,for,if,impl,
    in,let,loop,match,mod,move,mut,pub,ref,return,self,Self,static,struct,super,trait,true,type,unsafe,use,where,while
  },
  ndkeywords={i8, i16, i32, i64, u8, u16, u32, u64, usize, isize, f32, f64},
  ndkeywordstyle=\color{darkgray}\bfseries,
  identifierstyle=\color{black},
  sensitive=true, % keywords are case-sensitive
  comment=[l]{//}, % l is for line comment
  morecomment=[s]{/*}{*/}, % s is for start and end delimiter
  morestring=[b]" % defines that strings are enclosed in double quotes
}
%%%%%%%%%%%%%%%%%%%%%%%%%%%%%%%%%%%%%%%%%%%%%%%%%%%%%%%%%%%%%%%%%%%%%%%%%%%%%%%%
%% (Ende) Custom Syntax highlighting                                          %%
%%%%%%%%%%%%%%%%%%%%%%%%%%%%%%%%%%%%%%%%%%%%%%%%%%%%%%%%%%%%%%%%%%%%%%%%%%%%%%%%
\begin{document}
%% Set up title page, declaration of authorship, abstract, acknowledgements
\frontmatter
\makefrontmatter

%%%%%%%%%%%%%%%%%%%%%%%%%%%%%%%%%%%%%%%%%%%%%%%%%%%%%%%%%%%%%%%%%%%%%%%%%%%%%%%%
%% Danksagungen                                                               %%
%%%%%%%%%%%%%%%%%%%%%%%%%%%%%%%%%%%%%%%%%%%%%%%%%%%%%%%%%%%%%%%%%%%%%%%%%%%%%%%%
% \begin{acknowledgements}
%   Im Falle, dass Sie Ihrer Arbeit eine Danksagung für Ihre Unterstützer
%   (Familie, Freunde, Betreuer)
%   hinzufügen möchten, können Sie diese hier platzieren.

%   Dieser Part ist optional und kann im Quelltext auskommentiert werden.
% \end{acknowledgements}
%%%%%%%%%%%%%%%%%%%%%%%%%%%%%%%%%%%%%%%%%%%%%%%%%%%%%%%%%%%%%%%%%%%%%%%%%%%%%%%%
%% (Ende) Danksagungen                                                        %%
%%%%%%%%%%%%%%%%%%%%%%%%%%%%%%%%%%%%%%%%%%%%%%%%%%%%%%%%%%%%%%%%%%%%%%%%%%%%%%%%


\tableofcontents


\mainmatter

%%%%%%%%%%%%%%%%%%%%%%%%%%%%%%%%%%%%%%%%%%%%%%%%%%%%%%%%%%%%%%%%%%%%%%%%%%%%%%%%
%% Der Inhalt der Arbeit                                                      %%
%%                                                                            %%
%% Hier können Sie die schriftliche Ausarbeitung ihrer Arbeit                 %%
%% niederschreiben. Der Übersicht halber bietet sich jedoch an, dies in einer %%
%% oder mehreren separaten Dateien zu tun, welche mittels \input eingebunden  %%
%% werden --- wie auch in der Vorlage geschieht.                              %%
%%%%%%%%%%%%%%%%%%%%%%%%%%%%%%%%%%%%%%%%%%%%%%%%%%%%%%%%%%%%%%%%%%%%%%%%%%%%%%%%

%%%%%%%%%%%%%%%%%%%%%%%%%%%%%%%%%%%%%%%%%%%%%%%%%%%%%%%%%%%%%%%%%%%%%%%%%%%%%%%%
% Diese Datei beinhaltet den eigentlichen Inhalt Ihrer Arbeit.
%
% Es bietet sich der Übersicht halber an, die einzelnen Abschnitte jeweils
% in eigene Dateien zu schreiben und mittels \input einzubinden.
% Eine mögliche Verzeichnisstruktur sähe entsprechend so aus:
%
%     thesis/
%     +- tex/
%     |  +- introduction.tex
%     |  +- motivation.tex
%     |  +- experiments.tex
%     |  |  ...
%     |  +- conclusion.tex
%     +- abstract.tex
%     +- contents.tex
%     +- thesis.tex
%%%%%%%%%%%%%%%%%%%%%%%%%%%%%%%%%%%%%%%%%%%%%%%%%%%%%%%%%%%%%%%%%%%%%%%%%%%%%%%%

\section{Einleitung}

Dies ist der Hauptteil Ihrer Arbeit.
In der Datei \texttt{references.bib} finden Sie bereits einige Quellen,
die Sie wahrscheinlich zitieren mögen,
wie z.B. die B Methode~\cite{abrial1996b,abrial2010modeling}
oder \textsc{ProB}~\cite{leuschel2003prob,leuschel2008prob}.
Beachten Sie den Artikel ``Common Errors in Bibliographies'' von John Owens.%
\footnote{\url{https://www.ece.ucdavis.edu/~jowens/biberrors.html}}

\cref{sec:figures,sec:tables}
geben je ein kurzes Beispiel,
wie Bilder bzw. Tabellen in \LaTeX{} erstellt werden.
\cref{sec:plot} zeigt die Einbindung eines Graphen.


\subsection{Makefile}

Im Wurzelverzeichnis finden Sie ein \texttt{Makefile}.
Über das Terminal können Sie die folgenden Befehle aufrufen:


\begin{tabularx}{\textwidth}{lX}
  \toprule
  \texttt{make} & Kompiliert das PDF und löscht aux-Files. \\
  \texttt{make clean} & Löscht das PDF und dazugehörige aux-Files. \\
  \texttt{make bibtool} & Sortiert \texttt{references.bib}
  und formatiert die Einträge einheitlich. \\
  \texttt{make watch} & Rekompiliert das PDF bei Änderungen und
  hält die Anzeige in Ihrem PDF-Betrachter aktuell. \\
  \bottomrule
\end{tabularx}

\section{Notizen}%
\subsection{Bytecode}

Um die VM Cache-Effizient und damit schnell zu halten, ist es wichtig, ein günstiges Format für den internen Bytecode zu finden.
Ein Enum zu benutzen, würde bewirken, dass zur Laufzeit alle Instruktionen die selbe Größe hätten. Dies wäre jedoch nicht sinnvoll, da nur manche Instruktionen Parameter benötigen.
Es wäre eine Verschwendung von Arbeitsspeicher, wenn simple Add-Instruktionen die gleiche Größe hätten wie Call-Instruktionen.

Somit macht es Sinn, Instruktionen mit verschiedenen Größen zu verwenden. Die VM konsumiert eine Instruktion und dann abhängig von dieser eine beliebige Anzahl an Parametern.
Im Code ist dies mit einer Kombination aus Enums und Unions umgesetzt. Opcode ist ein Union, welches entweder eine Instruktion, ein Segment oder einen Acht-Bit Konstante sein kann. Die beiden Enums Instruction und Segment sind jeweils auch durch ein einziges Byte kodiert.

\begin{lstlisting}
  #[repr(u8)]
  pub enum Segment {
    Argument = 0,
    Local = 1,
    // ...
  }

  #[repr(u8)]
  pub enum Instruction {
    Add = 0,
    Sub = 1,
    // ...
  }

  #[repr(C)]
  pub union Opcode {
    instruction: Instruction,
    segment: Segment,
    constant: u8,
  }
\end{lstlisting}

Mit diesem System verbrauchen Instruktionen nur genau soviel Speicher wie notwendig, was eine effizientere Cache-Auslastung und somit bessere Performance ermöglicht.

\subsection{Parsing}
Da die Anwendung eine Reihe an Parsern enthält, macht es Sinn gemeinsame Komponenten auszulagern.

\subsection{Keyboard}
Das Keyboard handling ist eine Übersetzung des entsprechenden Java Codes in der offiziellen Implementierung.
Zunächst wurde dies in Javascript implementiert, jedoch war zu diesem Zeitpunkt schon eine Portierung in den Rust Code geplant.
Die Verarbeitung von Sonderzeichen, wie Backspace oder Enter ist in Javascript anders als in Java, da diese als String im Event stehen.


\section{Bilder und Co.}

\subsection{Bilder}%
\label{sec:figures}

In \cref{fig:initial-draft} ist festgehalten,
wie alles angefangen hat.


\subsection{Tabellen}%
\label{sec:tables}

\cref{table:truths} fasst die Wahrheiten dieser Welt zusammen.

\begin{table}[ht]
  \begin{center}
    \begin{tabular}{lr}
      \toprule
      Fakt                                & Wahrheitsgehalt \\
      \midrule
      booktabs Tabellen sind hübscher     & 90 \%           \\
      Han Solo schoss zuerst              & 100 \%          \\
      Game of Thrones fand ein gutes Ende & 0 \%            \\
      \bottomrule
    \end{tabular}
    \caption{Table of truths.}%
    \label{table:truths}
  \end{center}
\end{table}

\subsection{Plots}%
\label{sec:plot}

Sie können mithilfe von \texttt{tikz} und \texttt{pgfplots}
ganz leicht Graphen erstellen:

\begin{figure}[ht]
  \centering
  \begin{tikzpicture}
    \begin{axis}
      \foreach \y in {0,0.1,...,1} % Wiederholt \addplot mit jeweils anderem \y
        \addplot coordinates {
            ( 1, 3.0 -\y)
            ( 2, 3.25-\y)
            ( 3, 3.5 -\y)
            ( 4, 3.75-\y)
            ( 5, 4.0 -\y)
          };
    \end{axis}
  \end{tikzpicture}
  \caption{A beautiful plot.}%
  \label{fig:the-plot}
\end{figure}

\section{Conclusion}

Am Ende der Arbeit werden noch einmal die erreichten Ergebnisse
zusammengefasst und diskutiert.


%% Dieser Part kann auskommentiert werden, sollte kein Anhang nötig sein
% \appendix
\section{Zusätzliche Informationen}

Hier können Sie Ihren Anhang definieren.

Achten Sie darauf, dass der Anhang in Ihrer \texttt{thesis.tex}
initial auskommentiert ist.
Der entsprechende Part befindet sich nahe dem Ende der Datei.
Entfernen Sie bei Bedarf die Kommentierung um den Anhang nutzen zu können.


\section{Nutzung}

Der Anhang wird wie die Abschnitte des Hauptteils der Arbeit gestaltet,
also mit \texttt{\textbackslash section} Befehlen.

  \subsection{Unterabschnitte}
  Die Verwendung von Unterabschnitten im Anhang
  mittels \texttt{\textbackslash subsection}
  funktioniert ebenfalls!


%%%%%%%%%%%%%%%%%%%%%%%%%%%%%%%%%%%%%%%%%%%%%%%%%%%%%%%%%%%%%%%%%%%%%%%%%%%%%%%%
%% (Ende) Der Inhalt der Arbeit                                               %%
%%%%%%%%%%%%%%%%%%%%%%%%%%%%%%%%%%%%%%%%%%%%%%%%%%%%%%%%%%%%%%%%%%%%%%%%%%%%%%%%


\backmatter
\listoffigures
\listoftables

\clearpage
\bibliography{references}
%% Depending on Language, use german alphadin or original alpha
\iflanguage{ngerman}{
  \bibliographystyle{alphadin}
}{
  \bibliographystyle{alpha}
}

\end{document}
